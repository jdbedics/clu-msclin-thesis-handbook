\PassOptionsToPackage{unicode=true}{hyperref} % options for packages loaded elsewhere
\PassOptionsToPackage{hyphens}{url}
%
\documentclass[openany]{book}
\usepackage{lmodern}
\usepackage{amssymb,amsmath}
\usepackage{ifxetex,ifluatex}
\usepackage{fixltx2e} % provides \textsubscript
\ifnum 0\ifxetex 1\fi\ifluatex 1\fi=0 % if pdftex
  \usepackage[T1]{fontenc}
  \usepackage[utf8]{inputenc}
  \usepackage{textcomp} % provides euro and other symbols
\else % if luatex or xelatex
  \usepackage{unicode-math}
  \defaultfontfeatures{Ligatures=TeX,Scale=MatchLowercase}
\fi
% use upquote if available, for straight quotes in verbatim environments
\IfFileExists{upquote.sty}{\usepackage{upquote}}{}
% use microtype if available
\IfFileExists{microtype.sty}{%
\usepackage[]{microtype}
\UseMicrotypeSet[protrusion]{basicmath} % disable protrusion for tt fonts
}{}
\IfFileExists{parskip.sty}{%
\usepackage{parskip}
}{% else
\setlength{\parindent}{0pt}
\setlength{\parskip}{6pt plus 2pt minus 1pt}
}
\usepackage{hyperref}
\hypersetup{
            pdftitle={CLU MS Clinical Psychology Thesis Handbook},
            pdfauthor={Jamie Bedics, PhD, ABPP, Program Director},
            pdfborder={0 0 0},
            breaklinks=true}
\urlstyle{same}  % don't use monospace font for urls
\usepackage{longtable,booktabs}
% Fix footnotes in tables (requires footnote package)
\IfFileExists{footnote.sty}{\usepackage{footnote}\makesavenoteenv{longtable}}{}
\usepackage{graphicx,grffile}
\makeatletter
\def\maxwidth{\ifdim\Gin@nat@width>\linewidth\linewidth\else\Gin@nat@width\fi}
\def\maxheight{\ifdim\Gin@nat@height>\textheight\textheight\else\Gin@nat@height\fi}
\makeatother
% Scale images if necessary, so that they will not overflow the page
% margins by default, and it is still possible to overwrite the defaults
% using explicit options in \includegraphics[width, height, ...]{}
\setkeys{Gin}{width=\maxwidth,height=\maxheight,keepaspectratio}
\setlength{\emergencystretch}{3em}  % prevent overfull lines
\providecommand{\tightlist}{%
  \setlength{\itemsep}{0pt}\setlength{\parskip}{0pt}}
\setcounter{secnumdepth}{5}
% Redefines (sub)paragraphs to behave more like sections
\ifx\paragraph\undefined\else
\let\oldparagraph\paragraph
\renewcommand{\paragraph}[1]{\oldparagraph{#1}\mbox{}}
\fi
\ifx\subparagraph\undefined\else
\let\oldsubparagraph\subparagraph
\renewcommand{\subparagraph}[1]{\oldsubparagraph{#1}\mbox{}}
\fi

% set default figure placement to htbp
\makeatletter
\def\fps@figure{htbp}
\makeatother

\usepackage{booktabs}
\usepackage{amsthm}
\makeatletter
\def\thm@space@setup{%
  \thm@preskip=8pt plus 2pt minus 4pt
  \thm@postskip=\thm@preskip
}
\makeatother
\usepackage[]{natbib}
\bibliographystyle{apalike}

\title{CLU MS Clinical Psychology Thesis Handbook}
\author{Jamie Bedics, PhD, ABPP, Program Director}
\date{2020-05-16}

\begin{document}
\maketitle

{
\setcounter{tocdepth}{1}
\tableofcontents
}
\hypertarget{goal-of-the-handbook}{%
\chapter{Goal of the Handbook}\label{goal-of-the-handbook}}

\begin{center}\rule{0.5\linewidth}{0.5pt}\end{center}

The goal of this handbook is to provide students with the information needed to successfully complete the master's thesis in the MS in Clinical Psychology Program (MSCP) at California Lutheran University (CLU). The manual should be understood as a supplement to the broader policies and procedures defined by the program and university.

Updates can be found at: \url{https://jdbedics.github.io/thesishandbook/}

\begin{center}\rule{0.5\linewidth}{0.5pt}\end{center}

\hypertarget{scholarly-research}{%
\section{Scholarly Research}\label{scholarly-research}}

\begin{center}\rule{0.5\linewidth}{0.5pt}\end{center}

Scholarly research requires the skills of inquiry as a method of addressing a problem. As a participant in research activities in the MSCP program, students are expected to develop the following abilities:

\begin{enumerate}
\def\labelenumi{\arabic{enumi}.}
\tightlist
\item
  Create or contribute empirical knowledge to the existing body of information in a discipline.
\item
  Carryout systematic inquiry of a body of literature.
\item
  Use tools of research including analyzing existing research, implementing research designs, using and/or developing instrumentation, employing appropriate methods of data analysis, and handling the logistics of conducting a research study.
\item
  Work with faculty or other professionals on a research project.
\item
  Use scholarly writing techniques.
\end{enumerate}

Research conducted in the MSCP program may be a thesis or a research project. While both are important contributions to the body of knowledge in a discipline, they have different purposes as described in the following sections. The most critical being that the thesis is taken for credit and partial fulfillment of the degree and in replacement of the comprehensive exam.

\begin{center}\rule{0.5\linewidth}{0.5pt}\end{center}

\hypertarget{comprehensive-exam}{%
\section{Comprehensive Exam}\label{comprehensive-exam}}

By default, students entering the MSCP program are required to complete the comprehensive exam. Students can, however, choose to \emph{opt out} of the comprehensive exam and instead complete a thesis.

\textbf{What is the comprehensive exam?}

\begin{itemize}
\tightlist
\item
  A closed book essay test that covers all the material studied during the program.\\
\item
  The test is offered at the end of the spring semester during the second year.
\item
  The exam consists of a morning session (9AM-Noon) and an afternoon session (1PM-4PM).
\item
  During each session, students choose to respond to 3 of 5 questions.
\end{itemize}

\textbf{How the comprehensive exam \emph{replaces} the thesis}

\begin{itemize}
\tightlist
\item
  Students register for the exam by paying a \emph{comprehensive exam fee} during the semester they take the exam (typically spring of their 2nd year).
\item
  Students \textbf{do not} take \emph{PSYC 566 Thesis (3 units)} in the spring of their 2nd year.\\
\item
  As a result, students who take the comprehensive exams graduate with a total of \textbf{37-units}
  instead of 40-units with the thesis option.
\item
  Students who take the comprehensive exam are also \textbf{not} required to take \emph{PSYC 565 Research Practicum} in the fall of their 2nd year. Instead, students can choose an elective course in consultation with the Program Diretor.
\end{itemize}

\begin{center}\rule{0.5\linewidth}{0.5pt}\end{center}

\hypertarget{thesis}{%
\section{Thesis}\label{thesis}}

The thesis is the result of an original empirical investigation that creates new knowledge within a discipline. It solves a problem related to lack of knowledge and is generally composed of the following elements:

\begin{enumerate}
\def\labelenumi{\arabic{enumi}.}
\tightlist
\item
  Identification of a problem caused by lack of knowledge;
\item
  Background and literature review of existing information about the problem;
\item
  Methods to be used for obtaining the needed knowledge;
\item
  Resulting new knowledge;
\item
  Interpretation of the new knowledge;
\item
  Transparent and open sharing of materials and results via pre-registration and the open science framework.
\end{enumerate}

\textbf{Tasks:}

\begin{itemize}
\tightlist
\item
  Students complete all the requirements outlined in this manual.
\item
  Students enroll in \textbf{\emph{PSYC 565 Research Practicum (3 units)}} during the fall of their 2nd year.
\item
  Students enroll in \textbf{\emph{PSYC 566 Thesis (3 units)}} during the spring of their 2nd year.
\item
  Students who fail to complete the thesis by the end of the 2nd year can:

  \begin{enumerate}
  \def\labelenumi{\arabic{enumi}.}
  \tightlist
  \item
    take \textbf{\emph{PSYC 599-01 Thesis Continuation}} and pay all associated fees every semester until the thesis is completed or,
  \item
    take the comprehensive exam and pay the comprehensive exam fee.
  \end{enumerate}
\end{itemize}

\begin{center}\rule{0.5\linewidth}{0.5pt}\end{center}

\hypertarget{pros-and-cons-thesis-vs.-comps}{%
\section{Pros and Cons: Thesis Vs. Comps}\label{pros-and-cons-thesis-vs.-comps}}

\textbf{\emph{Thesis ``Pros''}}

\begin{itemize}
\tightlist
\item
  Students gain a high degree of expertise and mastery in the area under study.
\item
  The thesis timeline creates accountability and structure in completing the thesis.
\item
  Doctoral programs often look favorably towards a completed thesis that demonstrates students' ability to successfully complete a research project.
\item
  Doctoral programs that require a thesis might \emph{waive} the thesis requirement based upon the completed thesis at CLU.
\item
  Students can have the thesis bound into a book (see Thesis Binding).
\item
  Students can earn quality letters for recommendation from their thesis committee members. These letters are often more meaningful than letters from employers or course instructors.
\end{itemize}

\textbf{\emph{Thesis ``Cons''}}

\begin{itemize}
\tightlist
\item
  Despite the structure offered through coursework, the thesis requires a considerable amount of extra work and self-discipline. The amount of autonomy and work can be quite stressful.
\item
  Students take an extra 3-units (PSYC 566) in the spring of their 2nd year for a total of 40-units versus 37-units for the comprehensive exam option.
\end{itemize}

\begin{center}\rule{0.5\linewidth}{0.5pt}\end{center}

\textbf{\emph{Comps ``Pros''}}

\begin{itemize}
\tightlist
\item
  Students are given a review sheet to help them study.
\item
  The exam is completed in a single day compared to the thesis that takes 2-years.
\item
  Questions that are not adequately answered can be successfully remitted before ``failing.''
\item
  Students can still complete an \textbf{research project} (see next section) which would allow the first three \emph{thesis pros} to be achieved.
\item
  Students can choose any 3-credit elective during their second year (spring is recommended).
\item
  The entire program is 37-units versus 40-units with the thesis option.
\end{itemize}

\textbf{\emph{Comps ``Cons''}}

\begin{itemize}
\tightlist
\item
  A 6-hour, closed book, essay test can be stressful and exhausting.
\item
  Research experience is valued by PHD programs and many, but not all, PSYD programs.
\item
  The research project, if chosen, could not be as structured as the thesis option.
\end{itemize}

\begin{center}\rule{0.5\linewidth}{0.5pt}\end{center}

\hypertarget{research-project-option}{%
\section{Research Project Option}\label{research-project-option}}

\textbf{Research Project + Comprehensive Exam}

Students can complete their own independent research project, identical to the thesis, but without the coursework (PSYC 566) and obligation to follow the requirements in this manual. There are two scenarios where a student might choose to take the comprehensive exam and work on a research project :

\begin{enumerate}
\def\labelenumi{\arabic{enumi}.}
\item
  A student can decide, from the beginning of the program, that they want to avoid the pressure and extra work of the thesis requirements but use the program to work on an research project. They could take PSYC 565 in the fall of their second year but will not take PSYC 566 during the spring.
\item
  Students might attempt the thesis but, for a variety of reasons, fall behind and not be able to complete all the necessary requirements of the thesis in order to graduate. If this occurs, then the student can always move to the comprehensive exam (in order to graduate) while continuing to work on their thesis as an independent research project (for no credit).
\end{enumerate}

In both of these scenarios, students are required to take the comprehensive exam and pay the comprehensive exam fee in order to graduate and in in accordance with university policies which can be found through the \href{https://www.callutheran.edu/students/registrar/}{university registrar}.

\begin{center}\rule{0.5\linewidth}{0.5pt}\end{center}

\hypertarget{thesis-committee}{%
\chapter{Thesis Committee}\label{thesis-committee}}

\textbf{Finding a Committee}

For many faculty, being on a thesis committee is a lot of work and they are often reluctant to join a committee. Consequently, how students choose to approach faculty is very critical. Dr.~Bedics will walk students through much of the process.

\begin{center}\rule{0.5\linewidth}{0.5pt}\end{center}

\textbf{1. Identifying Potential Faculty}

Search the CLU website for potential faculty that could contribute to your idea. Look at faculty interests in graduate psychology, undergraduate psychology, as well as other departments at CLU. Students could also identify faculty at other universities that might serve on the committee. Once identified, students should email Dr.~Bedics to discuss \emph{prior to students contacting the faculty themselves.}

\begin{center}\rule{0.5\linewidth}{0.5pt}\end{center}

\textbf{2. Foot-in-the-Door}

Students first introduce themselves to faculty through email. \textbf{Do not ask them to join your committee} in the first email. Instead, ask them if they have time to answer questions about your project. If they don't have time or incentive to do this simple task then they will not be willing to be on your committee. The foot-in-the-door strategy allows students to gain feedback from the person even if they are not willing to be on students' committee.

Here is a properly formatted and professional email using the foot-in-the-door strategy:

\begin{verbatim}
Dear Dr. ###,

My name is STUDENT and I am currently developing my master’s thesis project at California Lutheran University in the MS Clinical Psychology Program.  The topic is <TOPIC>. I think your expertise in <AREA> would help me in thinking through some of the details of the thesis.  I was curious if you could make time to answer a few questions over email or perhaps even chat over the phone or during your office hours?  I have attached a brief summary of my project.

Thank you for your time and consideration.

Sincerely,

First Name Last Name
\end{verbatim}

If they reply, and are positive, then you it is your responsibility to be flexible with \emph{your} time. Also, please never ask for an appointment in the same week.

\begin{center}\rule{0.5\linewidth}{0.5pt}\end{center}

\textbf{3. Requesting Committee Membership} -- If the initial meeting goes well then Dr.~Bedics will email the faculty member to discuss the role of the chair and reader.

\begin{center}\rule{0.5\linewidth}{0.5pt}\end{center}

\textbf{4. Working with your committee}

The committee will be most successful when students establish clear expectations for meeting with their committee members. Clearly established expectations will prevent you from emailing \emph{too little} or \emph{too much}. It is best to suggest \emph{fewer} meetings, perhaps one at the beginning and one at the end of the semester. You will have more meetings with your chair than your reader. Establish the following:

\begin{itemize}
\item
  Make a clear statement that you are respectful of their time and do not want to meet too much or too little. Faculty often appreciate such direct and respectful statements.
\item
  Suggest two meetings a semester and go from there. They might suggest more or a timeline based on other criteria.
\end{itemize}

\begin{center}\rule{0.5\linewidth}{0.5pt}\end{center}

\hypertarget{thesis-checklist---overview}{%
\chapter{Thesis Checklist - Overview}\label{thesis-checklist---overview}}

Students who wish to pursue the thesis option are required to meet with Dr.~Bedics at the end of every semester in order to review their progress according to the following timeline. Students who miss any of the following steps are removed from the thesis option and will be required to complete the comprehensive exam in order to graduate.

\begin{longtable}[]{@{}lllll@{}}
\toprule
& Task & Date Due & Year & Finished\tabularnewline
\midrule
\endhead
1. & Thesis Topic Approved & October 1st & First Year & {[}\_\_\_\_\_{]}\tabularnewline
2. & Setup OSF & October 1st & First Year & {[}\_\_\_\_\_{]}\tabularnewline
3. & Literature Review Draft Psych 564 & December 15th & First Year & {[}\_\_\_\_\_{]}\tabularnewline
4. & Academic Good Standing & December 15th & First Year & {[}\_\_\_\_\_{]}\tabularnewline
5. & Method Section Draft & May 1st & First Year & {[}\_\_\_\_\_{]}\tabularnewline
6. & Literature Review Revision & May 1st & First Year & {[}\_\_\_\_\_{]}\tabularnewline
7. & Academic Good Standing & May 15th & First Year & {[}\_\_\_\_\_{]}\tabularnewline
8. & Committee Assignment & June 30th & Summer & {[}\_\_{]} Chair{[}\_\_{]} Reader\tabularnewline
9. & Academic Good Standing & July 3rd & Summer & {[}\_\_\_\_\_{]}\tabularnewline
10. & Enroll in PSYC 565 & August 1st & Second Year & {[}\_\_\_\_\_{]}\tabularnewline
11. & Committee Approval of Proposal & September 1st & Second Year & {[}\_\_\_\_\_{]}\tabularnewline
12. & IRB Submitted & November 1st & Second Year & {[}\_\_\_\_\_{]}\tabularnewline
13. & Academic Good Standing & December 15th & Second Year & {[}\_\_\_\_\_{]}\tabularnewline
14. & Enroll in PSYC 566 & December 15th & Second Year & {[}\_\_\_\_\_{]}\tabularnewline
15. & Complete Draft to Dr.~Bedics & May 1st & Second Year & {[}\_\_\_\_\_{]}\tabularnewline
16. & Committee Approval of Final & May 10th & Second Year & {[}\_\_{]} Chair{[}\_\_{]} Reader\tabularnewline
17. & OSF Approval & May 1st & Second Year & {[}\_\_\_\_\_{]}\tabularnewline
18. & Thesis Commons & May 15th & Second Year & {[}\_\_\_\_\_{]}\tabularnewline
19. & Thesis Binding & Optional & Second Year & {[}\_\_\_\_\_{]}\tabularnewline
20. & GitHub Blog & Optional & Second Year & {[}\_\_\_\_\_{]}\tabularnewline
21. & Shiny App & Optional & Second Year & {[}\_\_\_\_\_{]}\tabularnewline
\bottomrule
\end{longtable}

\hypertarget{thesis-topic-selection}{%
\section{Thesis Topic Selection}\label{thesis-topic-selection}}

\begin{center}\rule{0.5\linewidth}{0.5pt}\end{center}

\textbf{Defining the Problem Area}

The general thesis topic is required to be selected during the beginning of the first semester of the first year. The thesis topic, does not, however, determine the hypotheses, methodology or general approach taken by the student to understand the problem (e.g.~experimental, quasi-experimental, meta-analytic methods). It would be premature for any student to attempt to define a hypothesis in the first year when their understanding of the topic area is limited and not justified by a thorough literature review. Hypotheses are typically developed after the first year of study.

\begin{center}\rule{0.5\linewidth}{0.5pt}\end{center}

\textbf{\emph{Tips for finding a Topic}}

Students are often unnecessarily delayed in choosing a thesis topic area to study. The ambivalence can cause a significant delay in their progress and result in an inferior product. Students should keep a few ideas in mind:

\begin{enumerate}
\def\labelenumi{\arabic{enumi}.}
\item
  It certainly is helpful if you are \textbf{passionate} about the topic. When a person is passionate about a topic it means they naturally want to read and understand the area under study. They often go to sleep thinking about the topic and wake up already thinking about the topic and what they want to accomplish that day.
\item
  Passion often starts by examining \textbf{interest} and is acquired through hard work. It is okay if you are not passionate about any particular problem area. If you have an area that you are interested in then that's a good start and simple select one area and make that your focus.
\item
  \textbf{Select \emph{one} interest}. The problem is we all have multiple interests. You really like schizophrenia but at the same time you're really interested in forensic psychology and the prison system. One semester you might want to study psychosis and the next semester you might have more interest in the prison system. Students can ``flip flop'' not because of ambivanlence or some ``problem.'' They're both just interested to the student and they can't decide.
\end{enumerate}

Unfortunately, this process has two problems. First, it can assume there is a \emph{right} area to be studying and you should \emph{feel} one area more than another. Second, it delays the students ability to make a good project. Our advice is to stick with your decision regardless how you feel. Passion comes from hard work, lots of reading, lots of conversations especially during times when you are not \emph{feeling} it or doubtful. Most importantly, passion and expertise cannot come if you are not \textbf{focused} on one topic.

\begin{enumerate}
\def\labelenumi{\arabic{enumi}.}
\setcounter{enumi}{3}
\tightlist
\item
  \textbf{The thesis will not define you.} Student's should understand that the topic they select to study will not define their career. In this sense, the thesis topic itself is really not all that important. What is most important is that the students learns \emph{how} to become scholar and do research. Students cannot do that if they do not commit to the thesis topic early in the first semester. In sum, it would be nice if you are passionate or, at minimum, interested in the topic but it simply is not \textbf{\emph{necessary}} nor is it expected for you to get the most out of the master's program.
\end{enumerate}

\textbf{Due}: October 1st, first year.

\begin{center}\rule{0.5\linewidth}{0.5pt}\end{center}

\hypertarget{open-science-framework}{%
\section{Open Science Framework}\label{open-science-framework}}

\textbf{Creating a transparent and reproducible workflow}

\includegraphics{images/osf.png}

\href{https://osf.io/}{OSF} is a repository that allows researchers to transparently share their work with the larger scientific community. During the course of the program, students use OSF to organize their thesis and other independent research projects. Instructions for setting up an OSF project can be found \href{https://speakerdeck.com/jdbedics/osf-setup-and-class-project-introduction}{here} and will be reviewed with Dr.~Bedics at students' first advising meeting.

In addition to organizing students' workflow, OSF allows students to showcase their work to their peers and potential employers and doctoral advisors. Students in the thesis option are required to use OSF and it is strongly recommended for students completing the research project.

\textbf{Due}: October 1st, first year.

\begin{center}\rule{0.5\linewidth}{0.5pt}\end{center}

\hypertarget{literature-review-timeline}{%
\section{Literature Review Timeline}\label{literature-review-timeline}}

\begin{center}\rule{0.5\linewidth}{0.5pt}\end{center}

\textbf{Understanding the Problem}

The development of the literature review begins during the fall of the first year during PSYC 564 Advanced Research Methods. The literature review will become the ``introduction'' section of the final thesis paper. The literature review demonstrates the student's mastery of the literature surrounding the \emph{problem} to be addressed by the thesis. Initial drafts, such as that from PSYC 564, are 10-12 pages in length.

The development of the literature review is, however, ongoing throughout the two years of the program until the final draft is submitted on May 1st of the second year. The typical length of a \emph{complete} literature review is between \textbf{20-40 pages} long but there is no maximum length.

\textbf{Due}:

\begin{enumerate}
\def\labelenumi{\arabic{enumi}.}
\tightlist
\item
  December 15th, first year (First Major Draft);
\item
  May 15th, end of first year (Second Major Draft with Hypotheses);
\item
  May 1st, end of second year (Final Draft).
\end{enumerate}

\begin{center}\rule{0.5\linewidth}{0.5pt}\end{center}

\hypertarget{method-section-timeline}{%
\section{Method Section Timeline}\label{method-section-timeline}}

\begin{center}\rule{0.5\linewidth}{0.5pt}\end{center}

\textbf{Solving the Problem}

The method sections defines the procedures of the thesis. The method section consists of the participant selection, selection of methods of measurements or materials, the procedure and the data analytic method. The method section can be worked on in \emph{PSYC 552 Psychometrics} during spring of the first year and also \emph{PSYC 562 Statistics II: Regression.} The method section is finalized during \emph{PSYC 565 Research Practicum} in the fall of the second year.

\textbf{Due}:

\begin{enumerate}
\def\labelenumi{\arabic{enumi}.}
\tightlist
\item
  May 1st, first year (First Draft);
\item
  December 15th, second year (Second Draft)
\end{enumerate}

\begin{center}\rule{0.5\linewidth}{0.5pt}\end{center}

\hypertarget{committee-assignment}{%
\section{Committee Assignment}\label{committee-assignment}}

\begin{center}\rule{0.5\linewidth}{0.5pt}\end{center}

Committee members are faculty or experts in the field that support the students work on the thesis. Students work with the program director to find the most appropriate committee members to support their research project. The committee is composed of a Chair and a Reader. Their roles are described below as is the process for finding and selecting committee members.

\begin{center}\rule{0.5\linewidth}{0.5pt}\end{center}

\textbf{Committee Chair}

The chair must have \emph{content knowledge} of the area under investigation for the thesis. For example, if the thesis is on schizophrenia then the chair must have extensive knowledge of schizophrenia. The chair is either a part-time or full-time faculty member at CLU and is chosen with the approval of the Program Director, Dr.~Bedics. There can be exceptions to the above criteria with the approval of Dr.~Bedics.

\textbf{Committee Reader}

The reader must have either content knowledge or expert \emph{methodological knowledge} of the area under investigation for the thesis. For example, if the thesis is on schizophrenia and utilizes an experimental design the the reader can either have knowledge of schizophrenia \textbf{or} knowledge of the experimental methods proposed. The reader can be a part-time or full-time faculty member at CLU or a professional in the community with at least a Master's degree that has the aforementioned expertise. The reader must be chosen with approval of program advisor and thesis Committee Chair. There can be exceptions to the above criteria with the approval of Dr.~Bedics.

\begin{center}\rule{0.5\linewidth}{0.5pt}\end{center}

Please see the \textbf{Thesis Committee} section of this handbook for information on how to acquire a committee.

\textbf{Due}: June 30th, Summer after First Year

\begin{center}\rule{0.5\linewidth}{0.5pt}\end{center}

\hypertarget{committee-approval-of-proposal}{%
\section{Committee Approval of Proposal}\label{committee-approval-of-proposal}}

\begin{center}\rule{0.5\linewidth}{0.5pt}\end{center}

During the summer following the first year, committee members read the literature review and method section and provide a general statement of approval to Dr.~Bedics. Based upon this approval, students are allowed to progress to the \emph{thesis track.} The rest of the thesis process is guided through coursework including PSYC 565 Research Practicum in the fall of the second year and PSYC 566 Thesis in the spring of the second year.

\textbf{Due}: September 1st, Second Year

\begin{center}\rule{0.5\linewidth}{0.5pt}\end{center}

\hypertarget{academic-good-standing}{%
\section{Academic Good Standing}\label{academic-good-standing}}

\begin{center}\rule{0.5\linewidth}{0.5pt}\end{center}

Academic good standing refers to maintaining a GPA above a 3.0 throughout the entire program and acting consistently with all policies and procedures defined by the program (see Program Handbook) and university (see university policy and procedures). Any student who receives below a B- in any course is not allowed to complete the thesis for course credit and partial fulfillment of the degree. They can, however, complete a research project with the support of full-time faculty.

\textbf{Due}: Every semester

\begin{center}\rule{0.5\linewidth}{0.5pt}\end{center}

\hypertarget{coursework-relevant-to-the-thesis}{%
\chapter{Coursework Relevant to the Thesis}\label{coursework-relevant-to-the-thesis}}

\begin{center}\rule{0.5\linewidth}{0.5pt}\end{center}

The knowledge gained from every course taken at CLU can be used to improve the development of the thesis. For example, if you have an interest in examining a specific disorder for your thesis then it makes sense that you study that disorder in \emph{PSYC 510 Psychopathology}. In \emph{PSYC 560 Statistics I: Exploratory Data Analysis}, you might consider finding open data that allows you to better understand your problem area through data visualization.

There are, however, specific courses where the thesis is explicitly incorporated into course assignments. The following are the MSCP courses that explicitly incorporate elements of the thesis into the syllabi:

\begin{longtable}[]{@{}lllll@{}}
\toprule
\begin{minipage}[b]{0.03\columnwidth}\raggedright
PSYC\#\strut
\end{minipage} & \begin{minipage}[b]{0.27\columnwidth}\raggedright
Course\strut
\end{minipage} & \begin{minipage}[b]{0.26\columnwidth}\raggedright
Semester\strut
\end{minipage} & \begin{minipage}[b]{0.10\columnwidth}\raggedright
Year\strut
\end{minipage} & \begin{minipage}[b]{0.19\columnwidth}\raggedright
Task\strut
\end{minipage}\tabularnewline
\midrule
\endhead
\begin{minipage}[t]{0.03\columnwidth}\raggedright
564\strut
\end{minipage} & \begin{minipage}[t]{0.27\columnwidth}\raggedright
Adv. Research Methods\strut
\end{minipage} & \begin{minipage}[t]{0.26\columnwidth}\raggedright
Fall\strut
\end{minipage} & \begin{minipage}[t]{0.10\columnwidth}\raggedright
One\strut
\end{minipage} & \begin{minipage}[t]{0.19\columnwidth}\raggedright
Start Lit Review; Start References\strut
\end{minipage}\tabularnewline
\begin{minipage}[t]{0.03\columnwidth}\raggedright
562\strut
\end{minipage} & \begin{minipage}[t]{0.27\columnwidth}\raggedright
Statistics II: Regression\strut
\end{minipage} & \begin{minipage}[t]{0.26\columnwidth}\raggedright
Spring\strut
\end{minipage} & \begin{minipage}[t]{0.10\columnwidth}\raggedright
One\strut
\end{minipage} & \begin{minipage}[t]{0.19\columnwidth}\raggedright
Data Analytic Plan\strut
\end{minipage}\tabularnewline
\begin{minipage}[t]{0.03\columnwidth}\raggedright
552\strut
\end{minipage} & \begin{minipage}[t]{0.27\columnwidth}\raggedright
Psychometrics\strut
\end{minipage} & \begin{minipage}[t]{0.26\columnwidth}\raggedright
Spring\strut
\end{minipage} & \begin{minipage}[t]{0.10\columnwidth}\raggedright
One\strut
\end{minipage} & \begin{minipage}[t]{0.19\columnwidth}\raggedright
Method\strut
\end{minipage}\tabularnewline
\begin{minipage}[t]{0.03\columnwidth}\raggedright
521\strut
\end{minipage} & \begin{minipage}[t]{0.27\columnwidth}\raggedright
Ethics\strut
\end{minipage} & \begin{minipage}[t]{0.26\columnwidth}\raggedright
Summer\strut
\end{minipage} & \begin{minipage}[t]{0.10\columnwidth}\raggedright
One\strut
\end{minipage} & \begin{minipage}[t]{0.19\columnwidth}\raggedright
Pre-Registration\strut
\end{minipage}\tabularnewline
\begin{minipage}[t]{0.03\columnwidth}\raggedright
565\strut
\end{minipage} & \begin{minipage}[t]{0.27\columnwidth}\raggedright
Research Practicum\strut
\end{minipage} & \begin{minipage}[t]{0.26\columnwidth}\raggedright
Fall\strut
\end{minipage} & \begin{minipage}[t]{0.10\columnwidth}\raggedright
Two\strut
\end{minipage} & \begin{minipage}[t]{0.19\columnwidth}\raggedright
IRB, Intro, Method\strut
\end{minipage}\tabularnewline
\begin{minipage}[t]{0.03\columnwidth}\raggedright
566\strut
\end{minipage} & \begin{minipage}[t]{0.27\columnwidth}\raggedright
Thesis\strut
\end{minipage} & \begin{minipage}[t]{0.26\columnwidth}\raggedright
Spring\strut
\end{minipage} & \begin{minipage}[t]{0.10\columnwidth}\raggedright
Two\strut
\end{minipage} & \begin{minipage}[t]{0.19\columnwidth}\raggedright
Complete Draft due May 1st\strut
\end{minipage}\tabularnewline
\begin{minipage}[t]{0.03\columnwidth}\raggedright
599\strut
\end{minipage} & \begin{minipage}[t]{0.27\columnwidth}\raggedright
Thesis Continuation\strut
\end{minipage} & \begin{minipage}[t]{0.26\columnwidth}\raggedright
Every semester Post 2nd year\strut
\end{minipage} & \begin{minipage}[t]{0.10\columnwidth}\raggedright
Two+\strut
\end{minipage} & \begin{minipage}[t]{0.19\columnwidth}\raggedright
Every semester until the thesis is complete\strut
\end{minipage}\tabularnewline
\bottomrule
\end{longtable}

\textbf{PSYC 564 Advanced Research Methods}

In PSYC 564, students develop the beginning of their literature review and begin to master the 7th edition of the Publication Manual by the American Psychological Association (see formatting section).

\begin{center}\rule{0.5\linewidth}{0.5pt}\end{center}

\textbf{PSYC 562: Statistics II: Regression}

In this course, students learn the basics of statistical modeling with an emphasis on regression. Although the focus of the course is on regression, students can work with the instructor to consider models that might address their research hypotheses. If students have a published article that closely approximates the study they want to run then students might learn more about the analyses conducted in that article.

\begin{center}\rule{0.5\linewidth}{0.5pt}\end{center}

\textbf{PSYC 552: Psychometrics}

The emphasis on this course is psychometric theory which includes an understanding of the concepts of reliability and validity. Students can use this course to examining the psychometric properties of the most likely methods of measurement such as self-report questionnaires.

\begin{center}\rule{0.5\linewidth}{0.5pt}\end{center}

\textbf{PSYC 521: Clinical and Research Ethics}

In this course, students learn about the concept of open science which include reproducibility, transparency, and pre-registration. Students continue to develop their OSF page and learn how to pre-register their hypotheses.

\begin{center}\rule{0.5\linewidth}{0.5pt}\end{center}

\textbf{PSYC 565: Research Practicum}

In addition to finishing up their introduction and method sections, students main task is to complete the CLU IRB, \url{https://www.callutheran.edu/research/irb/}. The IRB process is very extensive and students are required to complete an online ethical training. Once the IRB is successfully passed, students can collect their data. The process of getting IRB approval can be a lengthy one and require several revisions. Students submit their IRB in collaboration with their committee chair who will submit the IRB proposal to the IRB committee on the student's behalf.

A critical part of the thesis is the power analysis. The power analysis determines whether or not the proposed study will be able to accomplish the proposed goals (i.e., adequately test the hypothesis(-es)). The most basic application of the power analysis is to determine the required sample size necessary to find the effect most likely to be achieved given prior work in the field. Students will complete a power analysis prior to submitting their IRB.

\begin{center}\rule{0.5\linewidth}{0.5pt}\end{center}

\textbf{PSYC 566: Thesis}

During this semester, students work independently on their thesis (with their committee members) and complete their results and discussion sections. The semester requires the student to structure their own time as their are no formal class meetings. Students are often collecting data during this final semester.

In order to pass PSYC 566, students are required to have a complete draft of their thesis that is approved by the chair, reader, and program director. Students often do not complete thesis for a variety of reasons including an inability to collect data, medical and health problems, or other life priorities. In these circumstances, students have two options:

\textbf{\emph{1. Continue the thesis}} - Students can choose to continue the thesis and enroll in PSYC 599, and pay the feel, every semester until the thesis is complete. Students are assigned the following grade for PSYC 566 until the thesis is approved:

\begin{itemize}
\tightlist
\item
  \textbf{``IP'' (In Progress)} is given for theses, practica, internships and courses wherein the work has been evaluated and found to be satisfactory to date, but the assignment of a grade must await its completion. ``IP'' carries no credit until replaced by a permanent grade. The ``IP'' grade may be replaced by the appropriate final letter grade within one calendar year from the start of the class. ``IP'' grades which have not been resolved will be changed to ``F'' (undergraduate) or ``NC'' (graduate) at the time the student's degree is posted.
\end{itemize}

\textbf{\emph{2. Drop the Thesis and take the Comprehensive Exam}} - Students can register for the comprehensive exam and pay the associated fee. The comprehensive exam is offered every semester and should be taken the first semester the student does not register for PSYC 599.

\begin{center}\rule{0.5\linewidth}{0.5pt}\end{center}

\textbf{PSYC 599: Thesis Continuation}

Students who do not complete the thesis by the end of their second year and choose to continue to work on thesis must register for PSYC 599 every semester until the thesis is approved.

\begin{center}\rule{0.5\linewidth}{0.5pt}\end{center}

\hypertarget{format-of-paper---overview}{%
\chapter{Format of Paper - Overview}\label{format-of-paper---overview}}

\includegraphics{images/apamanual.png}

The thesis paper is completed in a manner consistent with the \href{https://www.amazon.com/s?k=apa+publication+manual+7th+edition\&crid=7T10VJ2PYQZH\&sprefix=apa+pu\%2Caps\%2C261\&ref=nb_sb_ss_i_1_6}{Publication Manual of the APA (7th Edition)}.

\begin{itemize}
\item
  Title Page*
\item
  Signature Page*
\item
  Dedication*
\item
  Acknowledgements*
\item
  Table of Contents*
\item
  Abstract
\item
  Introduction
\item
  Method
\item
  Results
\item
  Discussion
\item
  References
\item
  Tables
\item
  Figures
\item
  Appendices
\item
  There are several sections that \textbf{do not} follow the 7th Edition of the Publication Manual:

  \begin{itemize}
  \tightlist
  \item
    Title Page
  \item
    Signature Page
  \item
    Dedication
  \item
    Acknowledgements
  \item
    Table of Contents
  \end{itemize}
\end{itemize}

\begin{center}\rule{0.5\linewidth}{0.5pt}\end{center}

\hypertarget{title-page}{%
\section{Title Page}\label{title-page}}

This page provides the name of the thesis project, names of the university and school or department, and date of completion. The title page should be prepared in accordance with the sample page found in this section. The date at the bottom of the page is the month and year the degree is awarded. The title page is unnumbered but is counted as page ``i.''

\includegraphics{images/titlepage.png}

\begin{center}\rule{0.5\linewidth}{0.5pt}\end{center}

\hypertarget{signature-page}{%
\section{Signature Page}\label{signature-page}}

This page provides the name of the author and blank lines for the signatures of the committee members and the Graduate Dean of the appropriate School. The pages are signed when the members and Dean determine that the thesis or project is complete. The approval page should comply with the page form found in Appendix B. It should bear original signatures for all copies. The date at the bottom of the page is the date the degree is awarded; however, the page is not counted in the numbering system.

\includegraphics{images/signaturepage.png}

\begin{center}\rule{0.5\linewidth}{0.5pt}\end{center}

\hypertarget{dedication-optional}{%
\section{Dedication (optional)}\label{dedication-optional}}

This optional page contains a brief dedication to the individual(s) whom the author wishes to honor. If included, this page is numbered as page ``ii'' (lower case Roman numeral).

\begin{center}\rule{0.5\linewidth}{0.5pt}\end{center}

\hypertarget{acknowledgements-optional}{%
\section{Acknowledgements (optional)}\label{acknowledgements-optional}}

This optional page lists persons and/ or institutions whom the author wishes to thank for their assistance in completing the thesis or project. Such assistance can be provision of personal, financial, or moral support, or access to data sets or subject populations. A brief statement as to the type of assistance provided may follow each person or institution named. If included, this page continues the lower case Roman numeral sequence begun above.

\begin{center}\rule{0.5\linewidth}{0.5pt}\end{center}

\hypertarget{table-of-contents}{%
\section{Table of Contents}\label{table-of-contents}}

\includegraphics{images/tablecontents.png}

\begin{center}\rule{0.5\linewidth}{0.5pt}\end{center}

\hypertarget{abstract-apa-style}{%
\section{Abstract (APA style)}\label{abstract-apa-style}}

The abstract follows APA style and typically completed in the spring of the second semester. A draft of the abstract, without results and without concluding statements, might be drafted during \emph{PSYC 565 Research Practicum} during the of the second year.

\begin{center}\rule{0.5\linewidth}{0.5pt}\end{center}

\hypertarget{introduction-apa-style}{%
\section{Introduction (APA style)}\label{introduction-apa-style}}

As noted in the section ``Literature Review Timeline,'' the introduction is drafted during the first semester of the first year in \emph{PSYC 564} and revised every semester of the year. The introduction is a comprehensive document that demonstrates the students' expertise in the area under study.

\begin{center}\rule{0.5\linewidth}{0.5pt}\end{center}

\hypertarget{method-apa-style}{%
\section{Method (APA style)}\label{method-apa-style}}

As noted in the ``Method Section Timeline,'' portions of the Method section can be drafted during the spring semester and finalized during PSYC 565 in the fall of the second year. The method section defines the procedures of the study and is required for successful IRB approval at CLU. IRB is accomplished during PSYC 565 in the of the second year but students should review CLU IRB requirements as soon as possible. A link to the CLU IRB can be found here: \url{https://www.callutheran.edu/research/irb/}.

\begin{center}\rule{0.5\linewidth}{0.5pt}\end{center}

\hypertarget{data-analytic-method-apa-style}{%
\section{Data Analytic Method (APA style)}\label{data-analytic-method-apa-style}}

The data analytic plan can be developed during PSYC 562 in the spring of the first year and finalized during PSYC 565 in the of the second year. At CLU, we exclusively use the statistical program language called R. The data analytic section will include a power analysis as well a link to the study's pre-registration of the data analysis.

\begin{center}\rule{0.5\linewidth}{0.5pt}\end{center}

\hypertarget{results-apa-style}{%
\section{Results (APA style)}\label{results-apa-style}}

The results are typically drafted during the spring of second year. \emph{PSYC 566 Thesis} is an independent study unit and there are no class meetings. Instead, students meet with their thesis chair to review and write their analyses in the results section.

\begin{center}\rule{0.5\linewidth}{0.5pt}\end{center}

\hypertarget{discussion-apa-style}{%
\section{Discussion (APA style)}\label{discussion-apa-style}}

The discussion is a critical section of the research paper and is typically \textbf{10-pages} minimum. In the discussion the student provide the following:

\begin{itemize}
\tightlist
\item
  A summary of the proposed hypotheses and rationale.
\item
  A review of the results in narrative form without the presentation of any statistics.
\item
  A discussion of the implications of the results for each hypothesis.
\item
  A discussion of future steps for each hypothesis.
\item
  A discussion of the limitations of the study in addressing each hypothesis.
\end{itemize}

The discussion section demonstrates the student's ability to place their study in the context of the larger literature and think more philosophically. It is a space for creativity and ingenuity.

\begin{center}\rule{0.5\linewidth}{0.5pt}\end{center}

\hypertarget{references-apa-style}{%
\section{References (APA style)}\label{references-apa-style}}

References follow APA style. Please make sure all in-text citations are in the reference section and all citations in the reference section are cited in the text.

\begin{center}\rule{0.5\linewidth}{0.5pt}\end{center}

\hypertarget{tables-figures-appendices-apa-style}{%
\section{Tables, Figures, \& Appendices (APA style)}\label{tables-figures-appendices-apa-style}}

Tables and Figures can be essential for helping a reader understand your data analysis and conclusions. Please follow APA formatting for all Table and Figures.

If students choose to bind their thesis then they pay careful attention to formatting in order to maintain the margins of the paper (1.5" all around). This can especially be a problem for Tables.

Appendices often include a student's curriculum vita, IRB approval form, and other materials.

\begin{center}\rule{0.5\linewidth}{0.5pt}\end{center}

\hypertarget{thesis-binding}{%
\chapter{Thesis Binding}\label{thesis-binding}}

The following are instructions for binding your thesis. The binding of your thesis is \emph{optional} and at your expense.You are responsible for the spelling, grammar, and correct APA formatting of your thesis. A bound thesis is a \textbf{final} thesis.

\begin{enumerate}
\def\labelenumi{\arabic{enumi}.}
\tightlist
\item
  At least three (3) bound copies of the Thesis must be ordered.

  \begin{enumerate}
  \def\labelenumii{\alph{enumii}.}
  \tightlist
  \item
    One copy for the Graduate School of Psychology, one copy for the Thesis Committee Chair, and one personal copy for your possession. You can order more if you prefer (see \#2).
  \item
    The three copies must be printed on 25\% rag or cotton fiber watermarked white paper, at least 20 pound weight, 8½ x 11 inches in size (EZERASE, or similar paper is not acceptable). A good example is Southworth Fine Business Paper, 25\% cotton, 24 pound, white, stock \#403C which is available for purchase from Office Depot, OfficeMax, and Staples. A similar 20 pound weight paper is also available.
  \item
    Original signed signature pages on the same paper must be submitted with each of the three copies.
  \end{enumerate}
\item
  Additional personal copies may be ordered at the same time.

  \begin{enumerate}
  \def\labelenumii{\alph{enumii}.}
  \tightlist
  \item
    Personal copies may be printed on paper of the student's choice (e.g., 20 pound paper).
  \item
    Signature pages for the personal copies may be photocopies of the originals as long as they are on paper that is identical to the rest of the thesis.
  \end{enumerate}
\item
  Copies for binding must be delivered to the Program Specialist.

  \begin{enumerate}
  \def\labelenumii{\alph{enumii}.}
  \tightlist
  \item
    The copies delivered to the Program Specialist are NOT to be bound - just packaged with bright colored paper separating the individual copies.
  \item
    Students are responsible for paying binding fees for all copies (the three required copies and for any additional personal copies). The cost is \$40 per copy (no matter the length), and to be paid by check to CLU. Prices may change.
  \item
    The Program Specialist will forward the copies to the bindery as they are delivered.
  \item
    Once the Program Specialist receives the copies and payment for binding, a change of grade will be submitted to the Registrar's Office.
  \end{enumerate}
\item
  The bound copies are typically ready in about 6-8 weeks and are distributed as follows:

  \begin{enumerate}
  \def\labelenumii{\alph{enumii}.}
  \tightlist
  \item
    The Graduate School of Psychology copy and the Thesis Committee Chair copy will be delivered via campus mail by the Program Specialist.
  \item
    Students will be notified when their personal copies are ready for pick-up.
  \end{enumerate}
\item
  If you have any questions regarding the binding process, please do not hesitate to contact Mengmeng Liu, Graduate Program Specialist, at 805-493-3662 or at \href{mailto:mengmengliu@callutheran.edu}{\nolinkurl{mengmengliu@callutheran.edu}}.
\end{enumerate}

\begin{center}\rule{0.5\linewidth}{0.5pt}\end{center}

\hypertarget{thesis-commons}{%
\chapter{Thesis Commons}\label{thesis-commons}}

\includegraphics{images/thesiscommons.png}

\href{https://thesiscommons.org/}{Thesis Commons} is a place for students to publish their thesis. Thesis Commons is supported by OSF and is a way to both archive and showcase your work along with your OSF project.

\hypertarget{presentations-and-publications}{%
\chapter{Presentations and Publications}\label{presentations-and-publications}}

The faculty hope you present your work at conferences and in publications. Presenting your work at conferences can be a very rewarding and fun experience. Typical conferences attended by CLU MSCP students inlcude:

\begin{itemize}
\tightlist
\item
  \href{https://westernpsych.org/convention/}{Western Psychological Association}
\item
  \href{https://convention.apa.org/}{American Psychological Association}
\item
  \href{https://www.psychologicalscience.org/conventions/annual}{Association for Psychological Science}
\end{itemize}

Faculty will also be happy to recommend various conferences given students specialized area of interest. Please remember to contact your chair \emph{prior} to submitting your work to any professional outlet. Your committee members will typically be authors on all of your publically published work.

\hypertarget{guidelines-and-procedures}{%
\chapter{Guidelines and Procedures}\label{guidelines-and-procedures}}

Successful completion of the thesis requires students to remain in good standing in accordance with the the guidelines regarding academic success and integrity set forth in the university's Graduate Catalog and the MS Clinical Psychology Handbook.

\begin{center}\rule{0.5\linewidth}{0.5pt}\end{center}

\textbf{Integrity}

Academic dishonesty diminishes the quality of scholarship and defrauds those who depend upon the integrity of the educational system. Consult the CLU Graduate Catalog for definitions of cheating, fabrication, facilitating academic dishonesty, and plagiarism. According to university policy, students who engage in academic dishonesty may be in jeopardy of disciplinary action, including suspension or expulsion from the university.

\begin{center}\rule{0.5\linewidth}{0.5pt}\end{center}

\textbf{Failure to Complete the Thesis}

Many students are unable to complete the thesis within the expected two years of study. Students who do not successfully complete the thesis within the first two years are required to maintain \textbf{continuous enrollment} by registering for \emph{PSYC 599C-01 Thesis Continuation} \textbf{every semester} until the thesis is successfully completed.

Under special circumstances, students can take a leave of absence from the university. Such a leave of absence is completed in accordance with university policies which can be found through the university \href{https://www.callutheran.edu/students/registrar/}{registrar}.

\begin{center}\rule{0.5\linewidth}{0.5pt}\end{center}

\bibliography{book.bib,packages.bib}

\end{document}
