\PassOptionsToPackage{unicode=true}{hyperref} % options for packages loaded elsewhere
\PassOptionsToPackage{hyphens}{url}
%
\documentclass[]{book}
\usepackage{lmodern}
\usepackage{amssymb,amsmath}
\usepackage{ifxetex,ifluatex}
\usepackage{fixltx2e} % provides \textsubscript
\ifnum 0\ifxetex 1\fi\ifluatex 1\fi=0 % if pdftex
  \usepackage[T1]{fontenc}
  \usepackage[utf8]{inputenc}
  \usepackage{textcomp} % provides euro and other symbols
\else % if luatex or xelatex
  \usepackage{unicode-math}
  \defaultfontfeatures{Ligatures=TeX,Scale=MatchLowercase}
\fi
% use upquote if available, for straight quotes in verbatim environments
\IfFileExists{upquote.sty}{\usepackage{upquote}}{}
% use microtype if available
\IfFileExists{microtype.sty}{%
\usepackage[]{microtype}
\UseMicrotypeSet[protrusion]{basicmath} % disable protrusion for tt fonts
}{}
\IfFileExists{parskip.sty}{%
\usepackage{parskip}
}{% else
\setlength{\parindent}{0pt}
\setlength{\parskip}{6pt plus 2pt minus 1pt}
}
\usepackage{hyperref}
\hypersetup{
            pdftitle={CLU MS Clinical Psychology Thesis Handbook},
            pdfauthor={Jamie Bedics},
            pdfborder={0 0 0},
            breaklinks=true}
\urlstyle{same}  % don't use monospace font for urls
\usepackage{longtable,booktabs}
% Fix footnotes in tables (requires footnote package)
\IfFileExists{footnote.sty}{\usepackage{footnote}\makesavenoteenv{longtable}}{}
\usepackage{graphicx,grffile}
\makeatletter
\def\maxwidth{\ifdim\Gin@nat@width>\linewidth\linewidth\else\Gin@nat@width\fi}
\def\maxheight{\ifdim\Gin@nat@height>\textheight\textheight\else\Gin@nat@height\fi}
\makeatother
% Scale images if necessary, so that they will not overflow the page
% margins by default, and it is still possible to overwrite the defaults
% using explicit options in \includegraphics[width, height, ...]{}
\setkeys{Gin}{width=\maxwidth,height=\maxheight,keepaspectratio}
\setlength{\emergencystretch}{3em}  % prevent overfull lines
\providecommand{\tightlist}{%
  \setlength{\itemsep}{0pt}\setlength{\parskip}{0pt}}
\setcounter{secnumdepth}{5}
% Redefines (sub)paragraphs to behave more like sections
\ifx\paragraph\undefined\else
\let\oldparagraph\paragraph
\renewcommand{\paragraph}[1]{\oldparagraph{#1}\mbox{}}
\fi
\ifx\subparagraph\undefined\else
\let\oldsubparagraph\subparagraph
\renewcommand{\subparagraph}[1]{\oldsubparagraph{#1}\mbox{}}
\fi

% set default figure placement to htbp
\makeatletter
\def\fps@figure{htbp}
\makeatother

\usepackage{booktabs}
\usepackage{amsthm}
\makeatletter
\def\thm@space@setup{%
  \thm@preskip=8pt plus 2pt minus 4pt
  \thm@postskip=\thm@preskip
}
\makeatother
\usepackage[]{natbib}
\bibliographystyle{apalike}

\title{CLU MS Clinical Psychology Thesis Handbook}
\author{Jamie Bedics}
\date{2020-05-04}

\begin{document}
\maketitle

{
\setcounter{tocdepth}{1}
\tableofcontents
}
\hypertarget{the-ms-clinical-psychology-handbook}{%
\chapter{The MS Clinical Psychology Handbook}\label{the-ms-clinical-psychology-handbook}}

The goal of this handbook is to provide students with the information needed to successfully complete the master's thesis in MS in Clinical Psychology Program (MSCP) at California Lutheran University. The manual should be understood as a supplement to the broader policies and procedures defined by the program and university.

\hypertarget{section}{%
\section{}\label{section}}

\hypertarget{thesis-checklist}{%
\chapter{Thesis Checklist}\label{thesis-checklist}}

Instruction: Students are required to meet with the Dr.~Bedics at the end of every semester to review the required material for that semester. Students who miss any of the following steps will be automatically removed from the thesis option and required to complete the comprehensive exam. The student can, however, complete a research project but not for the partial fulfillment of the degree (i.e., credit).

\begin{longtable}[]{@{}lllll@{}}
\toprule
& Task & Date Due & Year & Finished\tabularnewline
\midrule
\endhead
1. & Thesis Topic Approved & October 1st & First Year & {[}\_\_\_\_\_{]}\tabularnewline
2. & Literature Review Draft Psych 564 & December 15th & First Year & {[}\_\_\_\_\_{]}\tabularnewline
3. & Academic Good Standing & December 15th & First Year & {[}\_\_\_\_\_{]}\tabularnewline
4. & Method Section & May 1st & First Year & {[}\_\_\_\_\_{]}\tabularnewline
5. & Literature Review Revision & May 1st & First Year & {[}\_\_\_\_\_{]}\tabularnewline
6. & Academic Good Standing & May 15th & First Year & {[}\_\_\_\_\_{]}\tabularnewline
7. & Committee Assignment & June 30th & Summer & {[}\_\_{]} Chair{[}\_\_{]} Reader\tabularnewline
8. & Academic Good Standing & July 3rd & Summer & {[}\_\_\_\_\_{]}\tabularnewline
9. & Enroll in PSYC 565 & August 1st & Second Year & {[}\_\_\_\_\_{]}\tabularnewline
10. & Committee Approval of Proposal & October 1st & Second Year & {[}\_\_\_\_\_{]}\tabularnewline
11. & IRB Submitted & November 1st & Second Year & {[}\_\_\_\_\_{]}\tabularnewline
12. & Academic Good Standing & December 15th & Second Year & {[}\_\_\_\_\_{]}\tabularnewline
13. & Enroll in PSYC 566 & December 15th & Second Year & {[}\_\_\_\_\_{]}\tabularnewline
14. & Draft to Dr.~Bedics & May 1st & Second Year & {[}\_\_\_\_\_{]}\tabularnewline
15. & Committee Approval of Final & May 10th & Second Year & {[}\_\_{]} Chair{[}\_\_{]} Reader\tabularnewline
16. & OSF Approval & May 1st & Second Year & {[}\_\_\_\_\_{]}\tabularnewline
17. & Thesis Commons & May 15th & Second Year & {[}\_\_\_\_\_{]}\tabularnewline
18. & Thesis Binding & Optional & Second Year & {[}\_\_\_\_\_{]}\tabularnewline
19. & GitHub Blog & Optional & Second Year & {[}\_\_\_\_\_{]}\tabularnewline
20. & Shiny App & Optional & Second Year & {[}\_\_\_\_\_{]}\tabularnewline
\bottomrule
\end{longtable}

Several, but not all, of the above tasks are detailed more fully in the following sections.

\hypertarget{thesis-topic-approval---defining-the-problem-area-1}{%
\section{Thesis Topic Approval - ``Defining the Problem Area'' (\#1)}\label{thesis-topic-approval---defining-the-problem-area-1}}

The general thesis topic is required to be selected during the beginning of the first semester of the first year. The thesis topic, does not, however, determine the hypotheses, methodology or general approach taken by the student to understand the problem (e.g.~experimental, quasi-experimental, meta-analytic methods).

\hypertarget{literature-review---understanding-the-problem-2-5}{%
\section{Literature Review - ``Understanding the Problem'' (\#2, \#5)}\label{literature-review---understanding-the-problem-2-5}}

The development of the literature review on the selected topic begins during the fall of the first during PSYC 564 Advanced Research Methods. The literature review will become the ``introduction'' section of the final thesis project. The literature review shows the student's mastery of the literature. More importantly, the literature review shows the students understanding of the \emph{problem} to be addressed and answered by the methodology. The literature is to be worked on throughout the duration of the program.

\hypertarget{method-section---solving-the-problem-4}{%
\section{Method Section - ``Solving the Problem'' (\#4)}\label{method-section---solving-the-problem-4}}

The method sections defines the procedures of the thesis project. The method section consists of the participant selection, selection of methods of measurements or materials, and the procedure. The method section can be worked on in PSYC 552 Psychometrics during spring of the first

\hypertarget{committee-assignment-7}{%
\section{Committee Assignment (\#7)}\label{committee-assignment-7}}

Committee members are faculty or experts in the field that support the students work on the thesis. Students work with the program director to find the most appropriate committee members to support their research project.

\begin{itemize}
\tightlist
\item
  Students select \emph{2} committee members including a chair and a reader
\item
  One committee member must be affiliated with CLU; the second committee member can be from CLU or another institution
\item
  Committee members typically hold doctoral degrees in areas that support the students research
\item
  Students typically select a committee member who has domain expertise (often the chair) and one that has methodology expertise (reader).\\
\item
  All committee members are approved by Dr.~Bedics.
\end{itemize}

\hypertarget{committee-approval-of-proposal-10}{%
\section{Committee Approval of Proposal (\#10)}\label{committee-approval-of-proposal-10}}

During the summer following the first year, committee members read the literature review and method section and provide a general statement of approval to Dr.~Bedics. Based upon this approval, students are allowed to progress to the \emph{thesis track.} The rest of the thesis process is guided through coursework including PSYC 565 Research Practicum in the fall of the second year and PSYC 566 Thesis in the spring of the second year.

\hypertarget{academic-good-standing-12}{%
\section{Academic Good Standing (\#12)}\label{academic-good-standing-12}}

Academic good standing refers to maintaining a GPA above a 3.0 throughout the entire program and acting consistently with all policies and procedures defined by the program (see Program Handbook) and university (see university policy and procedures). Any student who receives below a B- in any course is not allowed to complete the thesis for course credit and partial fulfillment of the degree.

\hypertarget{formatting}{%
\chapter{Formatting}\label{formatting}}

\hypertarget{methods}{%
\chapter{Methods}\label{methods}}

We describe our methods in this chapter.

\hypertarget{applications}{%
\chapter{Applications}\label{applications}}

Some \emph{significant} applications are demonstrated in this chapter.

\hypertarget{example-one}{%
\section{Example one}\label{example-one}}

\hypertarget{example-two}{%
\section{Example two}\label{example-two}}

\hypertarget{final-words}{%
\chapter{Final Words}\label{final-words}}

We have finished a nice book.

\bibliography{book.bib,packages.bib}

\end{document}
